% THIS IS SIGPROC-SP.TEX - VERSION 3.1
% WORKS WITH V3.2SP OF ACM_PROC_ARTICLE-SP.CLS
% APRIL 2009
%
% It is an example file showing how to use the 'acm_proc_article-sp.cls' V3.2SP
% LaTeX2e document class file for Conference Proceedings submissions.
% ----------------------------------------------------------------------------------------------------------------
% This .tex file (and associated .cls V3.2SP) *DOES NOT* produce:
%       1) The Permission Statement
%       2) The Conference (location) Info information
%       3) The Copyright Line with ACM data
%       4) Page numbering
% ---------------------------------------------------------------------------------------------------------------
% It is an example which *does* use the .bib file (from which the .bbl file
% is produced).
% REMEMBER HOWEVER: After having produced the .bbl file,
% and prior to final submission,
% you need to 'insert'  your .bbl file into your source .tex file so as to provide
% ONE 'self-contained' source file.
%
% Questions regarding SIGS should be sent to
% Adrienne Griscti ---> griscti@acm.org
%
% Questions/suggestions regarding the guidelines, .tex and .cls files, etc. to
% Gerald Murray ---> murray@hq.acm.org
%
% For tracking purposes - this is V3.1SP - APRIL 2009
\documentclass{acm_proc_article-sp}

\makeatletter
\newif\if@restonecol
\makeatother
\let\algorithm\relax
\let\endalgorithm\relax
\usepackage{algorithm2e}

\begin{document}
\title{Analysis of Security-Cost Trade-off of Fully Homomorphic Encryption Schemes}

%
% You need the command \numberofauthors to handle the 'placement
% and alignment' of the authors beneath the title.
%
% For aesthetic reasons, we recommend 'three authors at a time'
% i.e. three 'name/affiliation blocks' be placed beneath the title.
%
% NOTE: You are NOT restricted in how many 'rows' of
% "name/affiliations" may appear. We just ask that you restrict
% the number of 'columns' to three.
% Because of the available 'opening page real-estate'
% we ask you to refrain from putting more than six authors
% (two rows with three columns) beneath the article title.
% More than six makes the first-page appear very cluttered indeed.
%
% Use the \alignauthor commands to handle the names
% and affiliations for an 'aesthetic maximum' of six authors.
% Add names, affiliations, addresses for
% the seventh etc. author(s) as the argument for the
% \additionalauthors command.
% These 'additional authors' will be output/set for you
% without further effort on your part as the last section in
% the body of your article BEFORE References or any Appendices.

\numberofauthors{2} %  in this sample file, there are a *total*
% of EIGHT authors. SIX appear on the 'first-page' (for formatting
% reasons) and the remaining two appear in the \additionalauthors section.
%
\author{
% You can go ahead and credit any number of authors here,
% e.g. one 'row of three' or two rows (consisting of one row of three
% and a second row of one, two or three).
%
% The command \alignauthor (no curly braces needed) should
% precede each author name, affiliation/snail-mail address and
% e-mail address. Additionally, tag each line of
% affiliation/address with \affaddr, and tag the
% e-mail address with \email.
%
% 1st. author
\alignauthor
Kais Chaabouni \\
       \affaddr{ENSIMAG}\\
       \affaddr{Grenoble INP}\\
       \affaddr{Grenoble, France}\\
       \email{kais.chaabouni@ensimag.imag.fr}
% 2nd. author
\alignauthor 
Amrit Kumar\\
       \affaddr{ENSIMAG-Ecole Polytechnique}\\
       \affaddr{Grenoble INP}\\
       \affaddr{Grenoble, France}\\
       \email{amrit.kumar@ensimag.imag.fr}
}

% There's nothing stopping you putting the seventh, eighth, etc.
% author on the opening page (as the 'third row') but we ask,
% for aesthetic reasons that you place these 'additional authors'
% in the \additional authors block, viz.

\date{30 July 1999}
% Just remember to make sure that the TOTAL number of authors
% is the number that will appear on the first page PLUS the
% number that will appear in the \additionalauthors section.

\maketitle

\begin{abstract}
We present a study on variants of fully homomorphic schemes and the trade-off cost/security of basic operations on bits, integers, strings... 
\end{abstract}

\keywords{ Fully Homomorphic Encryption, Security, Cost} % NOT required for Proceedings

\section{Introduction(1 page)}

Partial HME, Gentry's scheme from SWHME to FHE via bootstrapping. The purpose fo FHE, the protocl in general. Remote processing of computations on encrypted input transforms a native program into another program which has a higher circuit depth. The objective of this article is to analyse the security-cost tradeoff of remote computations. The implementations that exist and the organisation of the article. With citations to the referenced articles and papers.

\section{Fully homomorphic scheme(1 page)}

A general structure of the scheme.

\subsection{Two Variants of Fully Homomorphic Encryption Scheme}

A brief description of the two FHE schemes with their algorithmic complexity of the different stages in the scheme. 
-> Smart Vercautren
-> BGV

Discuss the security of these schemes.


\subsection{Considered Implementations}

The library used, version, implementaion details, versions, references, modifications required, how-to-installl, how-to-use.Platforms accepted and other depedeices.

\subsection{Intial set-up cost(The cost not depending on the actual program)}

Can be integrated with subsection 2.1 

\section{ Didactic example :  Maximum of two integer (1 page)}



\subsection{Input Program for Fully Homomorphic Encryption Scheme}

Details of the circuit or the program used to calculate the maximum of two bounded integers.


\subsection{Theoretical Cost with the Schemes}

Analyse the cost of the program wrt SV, BGV. 

\subsection{Experimental Analysis}

Once for all analysis of common operations(KeyGen, Encrypt, Decrypt of 1 bit) varying the algorithmic and security parameters.


Measurements taken are CPU time, Wall Time, Memory Usage by varying the input size  and the algorithmic parameters. Noise reduction threshold.

\paragraph{KeyGen}


\paragraph{Encryption and Decryption}

\section{Choice of Benchmarks}

3 different classes of problems : 
\begin{itemize}
\item Operations on bits,
 \item operation on integers (possibility of extension on blocks and comparision with RSA and 
 \item branching
\end{itemize}

For each problem we propse the program, analyse their cost and eventually on security.

Why certain other programs cannot be executed under homomorphic encryption.

\section{Evaluation on the Benchmarks (1 page)}

Measurements on varying the parameters or the plaintext space. Evaluation of program dependent properties. 

\section{Analysis of the results}

Anlayze the results previously obtained and its impact on security and complexity. Comparisions of security-cost tradeoff with other schemes like RSA.
\section{Conclusion}
Conclusion\\\\\\\\\\\\\\\\\\\\\\\\\\\\\\\\\\\\\\\\\\\\\\\\\\\\\\\\\\\\\
\bibliographystyle{abbrv}
\bibliography{}  % sigproc.bib is the name of the Bibliography in this case
% You must have a proper ".bib" file
%  and remember to run:
% latex bibtex latex latex
% to resolve all references
%
% ACM needs 'a single self-contained file'!
%
%APPENDICES are optional
%\balancecolumns
\balancecolumns
% That's all folks!
\end{document}
