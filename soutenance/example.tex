\documentclass{beamer}
%\documentclass[handout]{beamer}

\usepackage{ensislides}
\usepackage[utf8]{inputenc}
\usepackage[french]{babel}

\title[Version courte du titre]{Titre en version\\longue}

\subtitle{Un exemple} % (optional)

\author{Matthieu Moy}
% - Use the \inst{?} command only if the authors have different
%   affiliation.

\institute{Ensimag}

\date{2009-2010}

\begin{document}

\begin{frame}
  \titlepage
\end{frame}

\section{Début}

\subsection{Tout début}

\begin{frame} \frametitle{Utilisation d'ensislides}
  \begin{itemize}
  \item Récupérer l'archive sur \url{http://www-verimag.imag.fr/~moy/ensitools/}
  \item Aller dans le répertoire {\tt beamer/ensislides/}
  \item Regarder l'exemple.
  \end{itemize}
\end{frame}

\begin{frame} \frametitle{Compilation de slides}
  \begin{itemize}
  \item {\tt pdflatex example.tex}
    \pause
  \item {\tt pdflatex example.tex} (pour la table des matières)
  \end{itemize}
  \pause
  \question{Ça vous plait ?}
\end{frame}

\subsection{Suite après le début}

\begin{frame} \frametitle{Avertissement}
  \begin{itemize}
  \item \warning{C'est vert, mais ce n'est pas le style officiel
      Ensimag, bien sûr.}
  \end{itemize}
\end{frame}


\subsection{Suite après le début}

\begin{frame}
  \frametitle{Encore un slide}
\end{frame}

\section[Deux]{Deuxième partie}

\begin{frame}
  \frametitle{Encore un ...}
\end{frame}

\section[Trois]{Troisième partie}

\begin{frame}
  \frametitle{Et un dernier}
\end{frame}

\end{document}


%%% Local Variables: 
%%% mode: latex
%%% TeX-master: t
%%% End: 
