\documentclass{beamer}
%\documentclass[handout]{beamer}

\usepackage{ensislides}
\usepackage[utf8]{inputenc}
\usepackage[french]{babel}

\title[Fully Homomorphic Encryption Scheme]{Analyzing Security-Cost Trade-off for a\\Fully Homomorphic Encryption Scheme}

%\subtitle{Un exemple} % (optional)

%\author{Matthieu Moy}
\author{Amrit Kumar \\ Kais Chaabouni \\}
\date{\today}
% - Use the \inst{?} command only if the authors have different
%   affiliation.

\institute{Ensimag}
%\date{2012-2013}

\begin{document}

\begin{frame}
  \titlepage
\end{frame}

\section{Introduction}

\begin{frame}\frametitle{Introduction}
  \begin{itemize}
  \item What is FHE, what propreties it provides?\\
    
    \begin{itemize}
      \pause
    \item Partially HE: ex RSA: $c_1c_2=(m_1m_2)^e \ \textrm {mod}\ N$
      \pause
    \item $\mathcal{E}(x_1).\mathcal{E}(x_2)=\mathcal{E}(x_1.x_2)$ , $\mathcal{E}(x_1)+\mathcal{E}(x_2)=\mathcal{E}(x_1+x_2)$   
      
    \end{itemize}
    \pause
  \item Gentry's scheme and other schemes.  
    
    \begin{itemize}
      \pause
    \item ``Somewhat'' HE. \\ 
      \pause
      $r(c_1+c_2) < r(c_1)+r(c_2)$ , $r(c_1.c_2) < r(c_1).r(c_2)$
      \pause
    \item ``Bootsrapping'' \\ 
      \pause
      ``dirty'' ciphertext $\rightarrow$ ``clean'' ciphertext
      
    \end{itemize}
    \pause
  \item Implementations, Cost, Security. \\
    
    \begin{itemize}
      \pause
    \item \texttt{Min}, \texttt{Max} of $(a,b)$ on plaintext:\\
      \ \ \ \ \ \ \ \ \ \texttt{Max}= $a > b$? $a$ : $b$ \ \ , \ \ \texttt{Min}= $a \leq b$? $a$ : $b$ 
      \pause
      \question{ \texttt{Min}, \texttt{Max} of $(a,b)$ on ciphertext ?}
    \end{itemize}
  \end{itemize}
\end{frame}

\section{Theoretical Analysis}

\subsection{SV Scheme: cost-security of crypto-system}

\begin{frame} \frametitle{SV Scheme: cost-security of crypto-system}
  \begin{itemize}
  \item \texttt{KeyGen}

    %$F(x)=2^{2^n}$
    %\newline Do until p is prime\{ $S(x) \leftarrow_{R}B_{\infty , N}(\eta/2)$\;
    %\newline \phantom{x}\hspace{3ex} $G(x)=1+2.S(x)$ 
    %\newline \phantom{x}\hspace{3ex} $p= \mathrm{\textbf{resultant}}(G(x),F(x))$ \}
    %\newline $D(x)=\mathrm{\textbf{gcd}}(G(x),F(x))$ over $\mathbb{F}_p[x]$
    %\newline $\alpha$ $\leftarrow$ unique root of $D(x) \in \mathbb{F}_p $
    %\newline apply \texttt{xgcd} algorithm over $\mathbb{Q}[x]$ to obtain $Z(x).G(x)=p$ mod $F(x)$
    %\newline $B=z_0$ mod $2p$
    $pk = (p, \alpha)$ and $sk = (p , B)$
    \pause
  \item  \texttt{Encrypt}($m \in \{0,1\} , pk$): 
    \newline   $R(x) \leftarrow_{R}B_{\infty , N}(\mu/2)$
    \; $C(x)=m+2\cdot R(x)$ 
    \newline \phantom{x}\hspace{3ex} return  $c=C(\alpha)$ mod $p$;
    \newline \texttt{Decrypt}($c, sk$):
    \newline \phantom{x}\hspace{3ex} $m = (c - \lfloor c \cdot B/p \rceil )$ mod $2$;
    \newline \texttt{Add}($c_1$, $c_2$, $pk$):
    \newline \phantom{x}\hspace{3ex} $c_3=c_1+c_2$ mod $p$; 
    \newline \texttt{Mult}($c_1$, $c_2$, $pk$):
    \newline \phantom{x}\hspace{3ex}  $c_3=c_1.c_2$ mod $p$; 
    \pause
  \item security with parameters
  \item depth
  \end{itemize}
\end{frame}

\subsection{Programs on ciphertexts}

\begin{frame} \frametitle{Circuit transformation}
  \begin{itemize}
  \item \warning{C'est vert, mais ce n'est pas le style officiel
      Ensimag, bien sûr.}
  \item Circuit transformation
  \end{itemize}
\end{frame}


\begin{frame} 
 \frametitle{Theoretical cost}
  \begin{itemize}
  \item linear
  \item sorting
  \end{itemize}
\end{frame}

\section[Experimental Analysis]{Experimental Analysis}

\subsection{Security-Cost}

\begin{frame} \frametitle{titre3.1}
  \begin{itemize}
  \item cost of keyGen
  \item Cost depending on security parameters N, eta, mu of crypto system / program
  \item \warning{depth too little: wrong decryption}
  \item explication
  \end{itemize}
\end{frame}

\subsection{Implementations cost}

\begin{frame} \frametitle{titre3.2}
  \begin{itemize}
  \item figures
  \item explication: cout confirmé
  \end{itemize}
\end{frame}

\begin{frame} \frametitle{titre3.3}
  \begin{itemize}
  \item \warning{C'est vert, mais ce n'est pas le style officiel
      Ensimag, bien sûr.}
  \end{itemize}
\end{frame}

\section[Conclusion]{Conclusion}

\begin{frame}
  \frametitle{Conclusion}
  \begin{itemize}
  \item Transformation on programs (no jump nor branching)

  \item Cost of program

  \item Security and Cost

  \item Quality cryptosystem
  \end{itemize}
\end{frame}

\end{document}


%%% Local Variables: 
%%% mode: latex
%%% TeX-master: t
%%% End: 
