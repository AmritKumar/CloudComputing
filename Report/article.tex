% THIS IS SIGPROC-SP.TEX - VERSION 3.1
% WORKS WITH V3.2SP OF ACM_PROC_ARTICLE-SP.CLS
% APRIL 2009
%
% It is an example file showing how to use the 'acm_proc_article-sp.cls' V3.2SP
% LaTeX2e document class file for Conference Proceedings submissions.
% ----------------------------------------------------------------------------------------------------------------
% This .tex file (and associated .cls V3.2SP) *DOES NOT* produce:
%       1) The Permission Statement
%       2) The Conference (location) Info information
%       3) The Copyright Line with ACM data
%       4) Page numbering
% ---------------------------------------------------------------------------------------------------------------
% It is an example which *does* use the .bib file (from which the .bbl file
% is produced).
% REMEMBER HOWEVER: After having produced the .bbl file,
% and prior to final submission,
% you need to 'insert'  your .bbl file into your source .tex file so as to provide
% ONE 'self-contained' source file.
%
% Questions regarding SIGS should be sent to
% Adrienne Griscti ---> griscti@acm.org
%
% Questions/suggestions regarding the guidelines, .tex and .cls files, etc. to
% Gerald Murray ---> murray@hq.acm.org
%
% For tracking purposes - this is V3.1SP - APRIL 2009
\documentclass{acm_proc_article-sp}

\makeatletter
\newif\if@restonecol
\makeatother
\let\algorithm\relax
\let\endalgorithm\relax
\usepackage{algorithm2e}

\begin{document}
\title{Analysis of Security-Cost Trade-off of Fully Homomorphic Encryption Schemes}

%
% You need the command \numberofauthors to handle the 'placement
% and alignment' of the authors beneath the title.
%
% For aesthetic reasons, we recommend 'three authors at a time'
% i.e. three 'name/affiliation blocks' be placed beneath the title.
%
% NOTE: You are NOT restricted in how many 'rows' of
% "name/affiliations" may appear. We just ask that you restrict
% the number of 'columns' to three.
% Because of the available 'opening page real-estate'
% we ask you to refrain from putting more than six authors
% (two rows with three columns) beneath the article title.
% More than six makes the first-page appear very cluttered indeed.
%
% Use the \alignauthor commands to handle the names
% and affiliations for an 'aesthetic maximum' of six authors.
% Add names, affiliations, addresses for
% the seventh etc. author(s) as the argument for the
% \additionalauthors command.
% These 'additional authors' will be output/set for you
% without further effort on your part as the last section in
% the body of your article BEFORE References or any Appendices.

\numberofauthors{2} %  in this sample file, there are a *total*
% of EIGHT authors. SIX appear on the 'first-page' (for formatting
% reasons) and the remaining two appear in the \additionalauthors section.
%
\author{
% You can go ahead and credit any number of authors here,
% e.g. one 'row of three' or two rows (consisting of one row of three
% and a second row of one, two or three).
%
% The command \alignauthor (no curly braces needed) should
% precede each author name, affiliation/snail-mail address and
% e-mail address. Additionally, tag each line of
% affiliation/address with \affaddr, and tag the
% e-mail address with \email.
%
% 1st. author
\alignauthor
Kais Chaabouni \\
       \affaddr{ENSIMAG}\\
       \affaddr{Grenoble INP}\\
       \affaddr{Grenoble, France}\\
       \email{kais.chaabouni@ensimag.imag.fr}
% 2nd. author
\alignauthor 
Amrit Kumar\\
       \affaddr{ENSIMAG-Ecole Polytechnique}\\
       \affaddr{Grenoble INP}\\
       \affaddr{Grenoble, France}\\
       \email{amrit.kumar@ensimag.imag.fr}
}

% There's nothing stopping you putting the seventh, eighth, etc.
% author on the opening page (as the 'third row') but we ask,
% for aesthetic reasons that you place these 'additional authors'
% in the \additional authors block, viz.

\date{30 July 1999}
% Just remember to make sure that the TOTAL number of authors
% is the number that will appear on the first page PLUS the
% number that will appear in the \additionalauthors section.

\maketitle

\begin{abstract}
We present a study on variants of fully homomorphic schemes and the trade-off cost/security of basic operations on bits, integers, strings... 
\end{abstract}

\keywords{ Fully Homomorphic Encryption, Security, Cost} % NOT required for Proceedings

\section{Introduction(1 page)}
The homomorphic encryption allows few operations on encrypted data. Gentry was the first to establidh a fully homomorphic encryption scheme based on ideal lattices which allows operations on encrypted data
 but with a high cost. Since then many researches have been made trying to improve the schemes to establish more operations on encrypted data with less cost and better secutity in order to do computations on remote machines. 
In this article we study the implementation of basic classes of operations and different aspects of 
complexity-security, and we discuss the results of experimental measures based on several parameters.\\
The Gentry FHE protocol consists on ``somewhat homomorphic'' scheme, ``squashing'' and ``bootstrapping''. (explain later). 
There are variants of FHE schemes more optimized than Gentry's scheme, so we use in the experiments some open source implementations of these schemes.
\section{Fully homomorphic scheme(1 page)}

A general structure of the scheme.

\subsection{Two Variants of Fully Homomorphic Encryption Scheme}
A brief description of the two FHE schemes with their algorithmic complexity of the different stages in the scheme. 
-> Smart Vercauteren
-> BGV
Discuss the security of these schemes.



\begin{itemize}
\item Smart-Vercauteren scheme:\\
The scheme has 3 parameters: ($N, \nu, \mu$ ). The somewhat homomorphic scheme uses 5 algorithms: (\texttt{KeyGen}, \texttt{Encrypt}, \texttt{Decrypt}, \texttt{Add}, \texttt{Mult}}).
\begin{itemize}
\item \texttt{KeyGen}:
\begin{itemize}
\item $F(x)= x^{n}+1$
\item do
\begin{itemize}
\item $S(x)=_{R}(B_{\infty , N}(\nu/2)$
\item $G(x)=1+2.S(x)$
\item $p= resultant(G(x),F(x))$
\end{itemize}
until $p$ is prime
\item $D(x)=gcd(G(x),F(x))$ over $\mathbb{F}_p[x]$
\item $\alpha$: the unique root of $D(x)$
\item apply \texttt{fmpz\_poly\_xgcd} to obtain $Z(x)$ such that $Z(x).G(x)=p$ mod $F(x)$
\item $B=z_0$ mod $2p$
\item $PK = (p, \alpha)$ and $Sk = (p , B)$
\end{itemize}
\item \texttt{Encrypt}($m \in \{0,1\} , PK$):
\begin{itemize}
\item $R(x)=_{R}(B_{\infty , N}(\mu/2)$
\item $C(x)=m+2.R(x)$
\item $c=C(\alpha) mod p$
\item return $c$
\end{itemize}
\item \texttt{Decrypt}($c, SK$):
\begin{itemize}
\item $m= (c - round(c.B/p)) mod 2$
\item return $m$
\end{itemize}
\item \texttt{Add}($c_1$, $c_2$, $PK$):
\begin{itemize}
\item $c_3=c_1+c_2 mod p$
\item return $c_3$
\end{itemize}
\item \texttt{Mult}($c_1$, $c_2$, PK):
\begin{itemize}
\item $c_3=c_1.c_2 mod p$
\item return $c_3$
\end{itemize}
\end{itemize}
\item BGV:
\end{itemize}
\subsection{Considered Implementations}

The library used, version, implementaion details, versions, references, modifications required, how-to-installl, how-to-use.Platforms accepted and other depedeices.\\
For the Smart-Vercauteren implementation we use the library libScarab v1.0.0 which requires the following libraries: GMP, FLINT, MPIR, MPFR.
\subsection{Intial set-up cost(The cost not depending on the actual program)}

Can be integrated with subsection 2.1 

\section{ Didactic example :  Min-Max of two integers (1 page)}
Min-Max of two bounded integers of n-btis.
\subsection{Input Program for Fully Homomorphic Encryption Scheme}
\texttt{aIsGreater}($a=(a_{n-1},...,a_0}), b=(b_{n-1},...,b_0}$)):
\begin{itemize}
\item G= encrypt($0, PK$); Count= encrypt($1, PK$) \\
for $i \in $nbits-1 .. $0$
\begin{itemize}
\item tmp1= or($a_i.b_i , \neg(a_i).\neg(b_i)$)
\item tmp2= $a_i.\neg(b_i).$Count
\item Count= Count . tmp1
\item G= or(G, tmp2)
\end{itemize}
\end{itemize}
\texttt{min\_max}: 
\begin{itemize}
\item G= aIsGreater(a, b) 
\item max_i = or($a_i.G, b_i. \neg{G}$)
\item min_i = or($a_i.\neg{G}, b_i.G$)
\end{itemize}

\subsection{Theoretical Cost with the Schemes}

Analyse the cost of the program wrt SV, BGV.

The cost of this algorithm \texttt{aIsGreater} is:\\
 $2*encrypt+n*(3*encrypt+7*add+7*mul)$ .\\
The cost of this algorithm \texttt{min\_max} is:\\
$ encrypt+aIsGreater+2*n*(encrypt+3add+3mull)$\\
Thus this algorithm has $O(n^2)$ complexity. 

\subsection{Experimental Analysis}

Once for all analysis of common operations(KeyGen, Encrypt, Decrypt of 1 bit) varying the algorithmic and security parameters.


Measurements taken are CPU time, Wall Time, Memory Usage by varying the input size  and the algorithmic parameters. Noise reduction threshold.
\begin{itemize}
\item KeyGen\\
The measured time execution for KeyGen($N, \mu$)  for fixed values of parameters: $N=8$ and $\mu = 4$ shows a variation from 0.31 s to 13.41 s with mean = 2.726 s.\\
\item {Encryption and Decryption}\\


\item {Cost time of Min-Max program}\\
\begin{tabular}{|c|c|c|c|c|c|c|c|c|c|}
  \hline
  nbits  & 3 & 4 & 5 & 6 & 7 & 8 &  9  & 10 \\
  \hline
  time(s) & 4.17 & 5.84 & 7.18 & 4.29 & 4.65 & 5.9 & 6.71 & 6.73\\
  \hline
\end{tabular}
\end{itemize}
\section{Choice of Benchmarks}

3 different classes of problems : 
\begin{itemize}
\item Operations on bits,
 \item operation on integers (possibility of extension on blocks and comparision with RSA and 
 \item branching
\end{itemize}

For each problem we propse the program, analyse their cost and eventually on security.

Why certain other programs cannot be executed under homomorphic encryption.

\section{Evaluation on the Benchmarks (1 page)}

Measurements on varying the parameters or the plaintext space. Evaluation of program dependent properties. 

\section{Analysis of the results}

Anlayze the results previously obtained and its impact on security and complexity. Comparisions of security-cost tradeoff with other schemes like RSA.
\section{Conclusion}
Conclusion\\\\\\\\\\\\\\\\\\\\\\\\\\\\\\\\\\\\\\\\\\\\\\\\\\\\\\\\\\\\\
\bibliographystyle{abbrv}
\bibliography{}  % sigproc.bib is the name of the Bibliography in this case
% You must have a proper ".bib" file
%  and remember to run:
% latex bibtex latex latex
% to resolve all references
%
% ACM needs 'a single self-contained file'!
%
%APPENDICES are optional
%\balancecolumns
\balancecolumns
% That's all folks!
\end{document}
