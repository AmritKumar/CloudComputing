
\documentclass{acm_proc_article-sp}
\makeatletter
\newif\if@restonecol
\makeatother
\let\algorithm\relax
\let\endalgorithm\relax

\usepackage{listings}
\usepackage{algorithm2e}
\begin{document}
\title{Analysis of Security-Cost Trade-off of Fully Homomorphic Encryption Schemes}

%
% You need the command \numberofauthors to handle the 'placement
% and alignment' of the authors beneath the title.
%
% For aesthetic reasons, we recommend 'three authors at a time'
% i.e. three 'name/affiliation blocks' be placed beneath the title.
%
% NOTE: You are NOT restricted in how many 'rows' of
% "name/affiliations" may appear. We just ask that you restrict
% the number of 'columns' to three.
% Because of the available 'opening page real-estate'
% we ask you to refrain from putting more than six authors
% (two rows with three columns) beneath the article title.
% More than six makes the first-page appear very cluttered indeed.
%
% Use the \alignauthor commands to handle the names
% and affiliations for an 'aesthetic maximum' of six authors.
% Add names, affiliations, addresses for
% the seventh etc. author(s) as the argument for the
% \additionalauthors command.
% These 'additional authors' will be output/set for you
% without further effort on your part as the last section in
% the body of your article BEFORE References or any Appendices.

\numberofauthors{2} %  in this sample file, there are a *total*
% of EIGHT authors. SIX appear on the 'first-page' (for formatting
% reasons) and the remaining two appear in the \additionalauthors section.
%
\author{
\alignauthor
Kais Chaabouni \\
       \affaddr{ENSIMAG}\\
       \affaddr{Grenoble INP}\\
       \affaddr{Grenoble, France}\\
       \email{kais.chaabouni@ensimag.imag.fr}
% 2nd. author
\alignauthor 
Amrit Kumar\\
       \affaddr{ENSIMAG-Ecole Polytechnique}\\
       \affaddr{Grenoble INP}\\
       \affaddr{Grenoble, France}\\
       \email{amrit.kumar@ensimag.imag.fr}
}

% There's nothing stopping you putting the seventh, eighth, etc.
% author on the opening page (as the 'third row') but we ask,
% for aesthetic reasons that you place these 'additional authors'
% in the \additional authors block, viz.

\date{30 July 1999}
% Just remember to make sure that the TOTAL number of authors
% is the number that will appear on the first page PLUS the
% number that will appear in the \additionalauthors section.
\maketitle
\begin{abstract}
We present a study on variants of fully homomorphic schemes and the trade-off cost/security of basic operations on bits, integers, strings... 
\end{abstract}

\keywords{ Fully Homomorphic Encryption, Security, Cost} % NOT required for Proceedings

\section{Introduction(1 page)}
The homomorphic encryption allows few operations on encrypted data. Gentry was the first to establidh a fully homomorphic encryption scheme based on ideal lattices which allows operations on encrypted data
 but with a high cost. Since then many researches have been made trying to improve the schemes to establish more operations on encrypted data with less cost and better secutity in order to do computations on remote machines. 
In this article we study the implementation of basic classes of operations and different aspects of 
complexity-security, and we discuss the results of experimental measures based on several parameters.\\
The Gentry FHE protocol consists on ``somewhat homomorphic'' scheme, ``squashing'' and ``bootstrapping''. (explain later). 
There are variants of FHE schemes more optimized than Gentry's scheme, so we use in the experiments some open source implementations of these schemes.
\section{Fully homomorphic scheme(1 page)}
A general structure of the scheme.
\subsection{Two Variants of Fully Homomorphic Encryption Scheme}
A brief description of the two FHE schemes with their algorithmic complexity of the different stages in the scheme. 
-> Smart Vercauteren
-> BGV
Discuss the security of these schemes.



\begin{itemize}
\item Smart-Vercauteren scheme:\\
The scheme has 3 parameters: ($N, \nu, \mu$ ). The somewhat homomorphic scheme uses 5 algorithms: (\texttt{KeyGen}, \texttt{Encrypt}, \texttt{Decrypt}, \texttt{Add}, \texttt{Mult}}).
\begin{itemize}
\item \texttt{KeyGen}:
\begin{itemize}
\item $F(x)= x^{n}+1$
\item do
\begin{itemize}
\item $S(x)=_{R}(B_{\infty , N}(\nu/2)$
\item $G(x)=1+2.S(x)$
\item $p= resultant(G(x),F(x))$
\end{itemize}
until $p$ is prime
\item $D(x)=gcd(G(x),F(x))$ over $\mathbb{F}_p[x]$
\item $\alpha$: the unique root of $D(x)$
\item apply \texttt{fmpz\_poly\_xgcd} to obtain $Z(x)$ such that $Z(x).G(x)=p$ mod $F(x)$
\item $B=z_0$ mod $2p$
\item $PK = (p, \alpha)$ and $Sk = (p , B)$
\end{itemize}
\item \texttt{Encrypt}($m \in \{0,1\} , PK$):
\begin{itemize}
\item $R(x)=_{R}(B_{\infty , N}(\mu/2)$
\item $C(x)=m+2.R(x)$
\item $c=C(\alpha) mod p$
\item return $c$
\end{itemize}
\item \texttt{Decrypt}($c, SK$):
\begin{itemize}
\item $m= (c - round(c.B/p)) mod 2$
\item return $m$
\end{itemize}
\item \texttt{Add}($c_1$, $c_2$, $PK$):
\begin{itemize}
\item $c_3=c_1+c_2 mod p$
\item return $c_3$
\end{itemize}
\item \texttt{Mult}($c_1$, $c_2$, PK):
\begin{itemize}
\item $c_3=c_1.c_2 mod p$
\item return $c_3$
\end{itemize}
\end{itemize}
\item BGV:
\end{itemize}
\subsection{Considered Implementations}

The library used, version, implementaion details, versions, references, modifications required, how-to-installl, how-to-use.Platforms accepted and other depedeices.\\
For the Smart-Vercauteren implementation we use the library libScarab v1.0.0 which requires the following libraries: GMP, FLINT, MPIR, MPFR.
\subsection{Intial set-up cost(The cost not depending on the actual program)}

Can be integrated with subsection 2.1 

\section{ Didactic example :  Maximum of two integer (1 page)}



\subsection{Input Program for Fully Homomorphic Encryption Scheme}

Details of the circuit or the program used to calculate the maximum of two bounded integers.


\subsection{Theoretical Cost with the Schemes}

Analyse the cost of the program wrt SV, BGV. 

\subsection{Experimental Analysis}

Once for all analysis of common operations(KeyGen, Encrypt, Decrypt of 1 bit) varying the algorithmic and security parameters.


Measurements taken are CPU time, Wall Time, Memory Usage by varying the input size  and the algorithmic parameters. Noise reduction threshold.

\paragraph{KeyGen}


\paragraph{Encryption and Decryption}

\section{Choice of Benchmarks}

The benchmark functions and problems on which the implementations were evaluated are categorized into three classes : functions operating on bits, functions operating directly on integers or block of bits and functions where branching is required i.e \texttt{if-then-else} condition is evaluated. 

Functions operating on bits provide information on the tradeoff for calculations with bits, while when operating directly on integers, the tradeoff is obtained on size of integers. we note that the first two categories of functions can be evaluated using straight-line-programs. For others we choose evaluation of \texttt{if-then-else} condition on encrypted data  wich woukd make the evlaluation circuit longer than compared to the same evaluation on unencrypted data. The functions evaluated are : 

\begin{itemize}

\item XOR of bits
\item Sum of : 

	\begin{itemize}

		\item  bounded integers

		\item unbounded integers

 	\end{itemize}
\item Majority bit

\item sorting with the following methods :

	\begin{itemize}

		\item Insertion sort

		\item Odd-Even Merge sort

		\item Bitonic sort

	\end{itemize}

\end{itemize}

We include the program used to evalute the above functons and their cost in terms of \texttt{XOR} and \texttt{AND} gates. 

\textbf{XOR of $n$-bits :} \texttt{XOR} of $n$-bits requires $n-1$ \texttt{XOR} gates. 

\textbf{Sum of integers of $n$-bits :} Sum of two integers is performed using an $n$ \texttt{full-adder} circuits. The cost of a full adder circuit is $2$ \texttt{XOR} to calculate the sum and $2$ \texttt{AND}, $1$ \texttt{XOR} and $1$ \texttt{OR} for the carry. Hence the total complexity is $3n$ \texttt{AND}, $5n$ \texttt{XOR}. 

\textbf{Sum of arbitrary integers}

\textbf{Majority of $n$-bits :} Majority bit is evaluated by obtaining the sum of the bits.

Sorting is tested on a sorting network where comparator gates are used. Each comparator is designed using the following algorithm. 

\restylealgo{algoruled}

\linesnumbered

\begin{algorithm}[H]

 \SetVline

 \KwData{a:=$(a_0,a_1,\ldots, a_{n-1}$), b:=$(b_0,b_1,\ldots, b_{n-1})$ }

 \KwResult{Return Max(a,b) and Min(a,b)}

 $aIsGreater \leftarrow 0$\;
 $bitsEqual \leftarrow 1$\;

	
 \For{$i\leftarrow n-1 $ \KwTo $0$}{
						
     $aIsGreater \leftarrow (aIsGreater +\neg(b_i)a_i)* bitEqual$ \;
     $bitEqual \leftarrow bitEqual*(a_ib_i + \neg(a_i)*\neg(b_i))$		       	
 }

\For{$i\leftarrow 0$ \KwTo $n-1$}{

     $Max_i \leftarrow aIsGreater*a_i + \neg(aIsGreater)*b_i$ \;

     $Min_i \leftarrow a \neg(aIsGreater)*a_i + aIsGreater*b_i$ \;

 }



return($Max$, $Min$)

 \caption{Comparator\label{Code:algo}}

\end{algorithm}



Each comparator gate uses $n$ \texttt{AND} gates, $n$ \texttt{OR} gates and $n$ \texttt{NOT} gates to find the larger of the two bit sequences and then to regenerate the maximum and the minuimum $4n$ \texttt{AND} gates $2n$ \texttt{NOT} and \texttt{OR} gates i.e. $6n$ \texttt{AND} and \texttt{XOR} gates are required.  

\textbf{Insertion sort :} $n(n-1)/2$ comparator gates are used for insertion sort.

\textbf{Odd-Even Merge sort :} The following program is used to perform Odd-Even merge sort.

\lstset{                                    % line wrapping on
  language=C,
  frame=lines,
  captionpos=b
 }


\renewcommand{\lstlistingname}{Code}


\begin{figure}
\begin{lstlisting}[caption=Odd-Even Merge Sort]

 OddEvenMergeSort(int lo, int n)
if (n>1)
        {
            int m=n/2;
            oddEvenMergeSort(lo, m);
            oddEvenMergeSort(lo+m, m);
            oddEvenMerge(lo, n, 1);
        }
oddEvenMerge(int lo, int n, int r)
    {
        int m=r*2;
        if (m<n)
        {
	    //even subsequence ;		
            oddEvenMerge(lo, n, m); 
            //odd subsequence ;
	    oddEvenMerge(lo+r, n, m);    
            for (int i=lo+r; i+r<lo+n; i+=m)
                compare(i, i+r);
        }
        else
            compare(lo, lo+r);
    }


\end{lstlisting} 

\end{figure}


\textbf{Bitonic Sort}
\begin{figure}
\begin{lstlisting}[caption=Bitonic Sort]

sortup( int m, int n) {//from m to m+n
    if (n==1) return;
    sortup(m,n/2);
    sortdown(m+n/2,n/2);
    mergeup(m,n/2);
}
 sortdown(int m, int n) {//from m to m+n
    if (n==1) return;
    sortup(m,n/2);
    sortdown(m+n/2,n/2);
    mergedown(m,n/2);
}

mergeup(int m, int n) {
    if (n==0) return;
    
    for (i=0 : n) {
         compare(m+i,m+i+n);
    }
    mergeup(m,n/2);
    mergeup(m+n,n/2);
}
mergedown( int m,  int n) {
    if (n==0) return;
    for (i=0 : n) {
        compare(m+i,m+i+n);
    }
    mergedown(m,n/2);
    mergedown(m+n,n/2);
}
\end{lstlisting}
\caption{Bitonic Search}
\end{figure}
Measurements on varying the parameters or the plaintext space. Evaluation of program dependent properties. 




\section{Analysis of the results}

Anlayze the results previously obtained and its impact on security and complexity. Comparisions of security-cost tradeoff with other schemes like RSA.
\section{Conclusion}
Conclusion\\\\\\\\\\\\\\\\\\\\\\\\\\\\\\\\\\\\\\\\\\\\\\\\\\\\\\\\\\\\\
\bibliographystyle{abbrv}
\bibliography{}  % sigproc.bib is the name of the Bibliography in this case
% You must have a proper ".bib" file
%  and remember to run:
% latex bibtex latex latex
% to resolve all references
%
% ACM needs 'a single self-contained file'!
%
%APPENDICES are optional
%\balancecolumns
\balancecolumns
% That's all folks!
\end{document}
