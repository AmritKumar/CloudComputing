% THIS IS SIGPROC-SP.TEX - VERSION 3.1
% WORKS WITH V3.2SP OF ACM_PROC_ARTICLE-SP.CLS
% APRIL 2009
%
% It is an example file showing how to use the 'acm_proc_article-sp.cls' V3.2SP
% LaTeX2e document class file for Conference Proceedings submissions.
% ----------------------------------------------------------------------------------------------------------------
% This .tex file (and associated .cls V3.2SP) *DOES NOT* produce:
%       1) The Permission Statement
%       2) The Conference (location) Info information
%       3) The Copyright Line with ACM data
%       4) Page numbering
% ---------------------------------------------------------------------------------------------------------------
% It is an example which *does* use the .bib file (from which the .bbl file
% is produced).
% REMEMBER HOWEVER: After having produced the .bbl file,
% and prior to final submission,
% you need to 'insert'  your .bbl file into your source .tex file so as to provide
% ONE 'self-contained' source file.
%
% Questions regarding SIGS should be sent to
% Adrienne Griscti ---> griscti@acm.org
%
% Questions/suggestions regarding the guidelines, .tex and .cls files, etc. to
% Gerald Murray ---> murray@hq.acm.org
%
% For tracking purposes - this is V3.1SP - APRIL 2009
\documentclass{acm_proc_article-sp}

\makeatletter
\newif\if@restonecol
\makeatother
\let\algorithm\relax
\let\endalgorithm\relax
\usepackage{algorithm2e}

\begin{document}
\title{Analysis of Security-Cost Trade-off of Fully Homomorphic Encryption Schemes}

%
% You need the command \numberofauthors to handle the 'placement
% and alignment' of the authors beneath the title.
%
% For aesthetic reasons, we recommend 'three authors at a time'
% i.e. three 'name/affiliation blocks' be placed beneath the title.
%
% NOTE: You are NOT restricted in how many 'rows' of
% "name/affiliations" may appear. We just ask that you restrict
% the number of 'columns' to three.
% Because of the available 'opening page real-estate'
% we ask you to refrain from putting more than six authors
% (two rows with three columns) beneath the article title.
% More than six makes the first-page appear very cluttered indeed.
%
% Use the \alignauthor commands to handle the names
% and affiliations for an 'aesthetic maximum' of six authors.
% Add names, affiliations, addresses for
% the seventh etc. author(s) as the argument for the
% \additionalauthors command.
% These 'additional authors' will be output/set for you
% without further effort on your part as the last section in
% the body of your article BEFORE References or any Appendices.

\numberofauthors{2} %  in this sample file, there are a *total*
% of EIGHT authors. SIX appear on the 'first-page' (for formatting
% reasons) and the remaining two appear in the \additionalauthors section.
%
\author{
% You can go ahead and credit any number of authors here,
% e.g. one 'row of three' or two rows (consisting of one row of three
% and a second row of one, two or three).
%
% The command \alignauthor (no curly braces needed) should
% precede each author name, affiliation/snail-mail address and
% e-mail address. Additionally, tag each line of
% affiliation/address with \affaddr, and tag the
% e-mail address with \email.
%
% 1st. author
\alignauthor
Kais Chaabouni \\
       \affaddr{ENSIMAG}\\
       \affaddr{Grenoble INP}\\
       \affaddr{Grenoble, France}\\
       \email{kais.chaabouni@ensimag.imag.fr}
% 2nd. author
\alignauthor 
Amrit Kumar\\
       \affaddr{ENSIMAG-Ecole Polytechnique}\\
       \affaddr{Grenoble INP}\\
       \affaddr{Grenoble, France}\\
       \email{amrit.kumar@ensimag.imag.fr}
}

% There's nothing stopping you putting the seventh, eighth, etc.
% author on the opening page (as the 'third row') but we ask,
% for aesthetic reasons that you place these 'additional authors'
% in the \additional authors block, viz.

\date{30 July 1999}
% Just remember to make sure that the TOTAL number of authors
% is the number that will appear on the first page PLUS the
% number that will appear in the \additionalauthors section.

\maketitle
\tableofcontents

\begin{abstract}
We present a study on variants of fully homomorphic schemes and the trade-off cost/security of basic operations on bits, integers, strings... \\\\\\\\\\\\\\\\\\\\\\\\\\
\end{abstract}

\keywords{ Fully Homomorphic Encryption, Security, Cost} % NOT required for Proceedings

\section{Introduction}

(Motivation and significance of fully homomorphic encryption scheme such as Gentry's lattice-based scheme.  
This work evaluates the efficiency of FHE implementations for operations varying from XOR to sort on remote machines. Trade-off between cost and security is analyzed.) \\\\\\
A partially homomorphic encryption allows certain operations on ciphertexts such as addition or multiplication which implies a known operation on plaintext like RSA Encryption
\[\texttt{E}(\prod_{i=1}^{n}{a_i}) = \prod_{i=1}^{n}{\texttt{E}(a_i)}\] \\
then Gentry proved that there is a fully homomorphic encryption which allows both addition and multiplication on encrypted bits: 
\[\texttt{FHE}(m1) + \texttt{FHE} (m2) = \texttt{FHE}(m1+m2) \] 
\[\texttt{FHE}(m1) \times \texttt{FHE} (m2) = \texttt{FHE}(m1 \times m2) ;\ \ \ m1, m2 \in \left\{ 0, 1 \right\}\] 
\\\\\\\\\\\\\\\\\\\\\\\\\\\\\\\\\\\\\\\\\\\\\\\\\\\\\\\\\\\\\\\\\\\\\\\\\\\\\\\\\\\\\\\\\\\\\\\\\\\\\\
\section{Gentry's fully homomorphic scheme}
 FHE\\\\\\\
\subsection{The Framework}
<<<<<<< HEAD
(Overview of the algorithm and its complexity. )
In this study we use two variants of fully homomorphic encryption: 
\begin{itemize}
\item Smart-Vercauteren implementation


\item Gentry-Halevi-Smart implementation

\end{itemize}

 \\\\\\\\\\\\\\\\\\\\\\\\\\\\\\\\\\\\\\\\\\\\\\\\\\\\\\\\
=======
Overview of the algorithm and its complexity. 
>>>>>>> 63705b936d6f17bf9d3ada88740fc95eb1fc1fe1

\subsection{Example}
(Description using a typical operation like multiplication of two $n$-bit integers. Comparison with partial homomorphic encryption scheme such as RSA.)\\
The multipliction of two FHE-encrypted integers of n bits consists on multiplications and additions on their FHE-encrypted bits: \[(\texttt{a}_{n},\texttt{a}_{n-1}, ... ,\texttt{a}_{1}) . (\texttt{b}_{n},\texttt{b}_{n-1}, ... ,\texttt{b}_{1}) = (\texttt{r}_{2n},\texttt{r}_{2n-1}, ... ,\texttt{r}_{1})\] \\
We do $ n^{2}$ multiplications $ a_i.a_j$ and $ 2.(n + (n-1)(n-2)$ additions

\[r_1= a_1.b_1 ; c_{1,1}=0\] 
\[(r_2,c_{1,2})=\texttt{add}(a_2.b_1 ,a_1.b_2 ,c_{1,1}  ) \] 
\[(r_3,c_{2,3})=\texttt{add}(a_1.b_3,\texttt{add}( a_3.b_1, a_2.b_2 , c_{1,2} ))   \] 
 \[ \vdots    \] 
\[ (r_{n-1},c_{n-1,n-1})= ...    \] 
\[(r_{n},c_{n,n})=\texttt{add}(a_1.b_n, ...)      \] 
\[  (r_{n+1},c_{n+1,n+1})= ..   \] 
\[  \vdots      \] 
\[  (r_{2n-1},c_{2n-1,2n-1})=     \] 
\[ (r_{2n},c_{2n,2n})=  \texttt{add}(a_n.b_n, c_{2n,2n-1})    \]
So it has $O(n^{2})$ complexity. In addition to that there is the cost of key generation, encryption and decryption, and there is also the recrypt algorithm called after each addition or multiplication. On the other hand the RSA-Encryption is much more easier and has a O(1) complexity for the multiplication of n-bits integer.
\\\\\\\\\\\\\\\\\\\\\\\\\\\\\\\\\\\\\\\\\\\\\\\\\\
\\\\\\\\\\\\\\\\\\\\\\\\\\\\\\\\\\\\\\\\\\\\\\\\

\section{Experimental Evaluation of Cost}

(Describe the FHE algorithm and its corresponding implementation choice. Installation, set-up procedure and usage of the implementation. We detail the measurements to be taken, like Wall time, CPU time and memory usage based on certain security or algorithmic parameters.

Statistical analysis of the measurements using minimum, max, mean and standard deviation.) \\\\\\\\

In this section we will experimentally measure time cost of Xor-bits operations using two versions of fully homomorphic encryption programs: libScarab-1.0.0 implementing Smart-Vercauteren algorithm and HElib implementing Gentry-Halevi-Smart algorithm.
\\\\\\\\\\\\\\\\\\\\\\\\\\\\\\\\\\\\\\\\\\\\\\\\\\\\\\\\\\\\\\\\\\\\\\\\\\\\\\\\\\\\\\\\\\\\\\\\\\\\\\\\\\\\\\\\\\\\\\\\\\\\\\\\\\\\\\\\\\\\\\\\\\\\\\\\\\

\section{Evaluation on Bench Marks}

We describe the system parameters and test vectors. This section also includes description of the process of evaluation for the following operations :


\subsection{Operations}

The \textsc{Hcrypt} implementation provides two gates : \texttt{XOR} and \texttt{AND} forming a complete set of functional gates. We hence define $\neg$\texttt{a} as \texttt{XOR(a,1)}, while \texttt{OR(a,b)} as \texttt{XOR(XOR(a,b), AND(a,b))}, where \texttt{a,b} are bits. 

Using these gates we build and analyse the cost of the following operations.

\paragraph{XOR of $n$-bits}
We start with the simplest operations on bits. \texttt{XOR} of $n$-bits requires $n-1$ \texttt{XOR} gates. 
\paragraph{Majority of $n$-bits} 
Majority bit is evaluated with the following circuit : 

\[\prod_{i=1}^{n}{a_i}+\sum_{i=1}^{n}{(\prod_{j!=i}{a_j}) \neg a_i}\]

where $\sum$ is the successive \texttt{OR}, and $\prod$ is the successive application of \texttt{AND}.

Hence it uses $(n^2-1)$ \texttt{AND} gates, and $n$ \texttt{OR} and \texttt{NOT} gates i.e. $(n^2+n-1)$ \texttt{AND}, $3n$ \texttt{XOR} gates.
\paragraph{Sum of integers of $n$-bits} Sum of two integers can be performed using an $n$-bit adder based on \texttt{full-adder} circuits. The cost of a full adder circuit is $2$ \texttt{XOR} to calculate the sum and $2$ \texttt{AND}, $1$ \texttt{XOR} and $1$ \texttt{OR} for the carry. Hence the total complexity is $3n$ \texttt{AND}, $5n$ \texttt{XOR}. 
\paragraph{Sum of arbitrary integers}

\begin{itemize}
\item Sum of : 
	\begin{itemize}
	\item  bounded integers
	\item unbounded integers
 	\end{itemize}
\item String sorting with the following methods :
	\begin{itemize}
	\item Insertion sort
	\item Merge sort
	\item Quick Sort
	\end{itemize}
\item Matrix product
\end{itemize}


Sorting is tested on a sorting network where comparator gates are used. Each comparator is designed using the following algorithm. 

\restylealgo{algoruled}
\linesnumbered
\begin{algorithm}[H]
 \SetVline
 \KwData{$(a_0,a_1,\ldots, a_{n-1}$), $(b_0,b_1,\ldots, b_{n-1})$ }
 \KwResult{Return Max(a,b) and Min(a,b)}
 aIsGreater=$0$\;

 \For{$i\leftarrow n-1$ \KwTo $0$}{
     $aIsGreater=aIsGreater +\neg(b_i)a_i$ \;
 }

\For{$i\leftarrow 0$ \KwTo $n-1$}{
     $Max_i = aIsGreater*a_i + \neg(aIsGreater)*b_i$ \;
     $Min_i = \neg(aIsGreater)*a_i + aIsGreater*b_i$ \;
 }

return($Max$, $Min$)
 \caption{Comparator\label{Code:algo}}
\end{algorithm}

Each comparator gate uses $n$ \texttt{AND} gates, $n$ \texttt{OR} gates and $n$ \texttt{NOT} gates to find the larger of the two bit sequences and then to regenerate the maximum and the minuimum $4n$ \texttt{AND} gates $2n$ \texttt{NOT} and \texttt{OR} gates i.e. $6n$ \texttt{AND} and \texttt{XOR} gates are required.  

\paragraph{Insertion sort} $n(n-1)/2$ comparator gates are use for insertion sort.

For each operation, we provide the initial algorithm and the results obtained. Observations based on the measurements are described. 
\\\\\\\\\\\\\\\\\\\\\\\\\\\\\\\\\\\\\\\\\\\\\\\\\\\\\\\\\\\\\\\\\\\\\\\\\\\\\\\\\\\\\\\\\\\\\\\\\\\\\\\\\\\\\\\\\\\\\\\\\\\\\\\\\\\\\\\\\\\\\\\\\\\\\\\\\\\\\\\\\\\\\\\\\\\\\\\\\\\\\\\\\\\\\\\\\\\\\\\\\\\\\\\\\\\\\\\\\\\\\\\\\\\\\\\\\\\\\\\\\\\\\\\\\\\\\\\\\\\\\\\\\\\\\\\\\\\\\\\\\\\\\\\\\\\\\\\\\\\\\\\\\\\\\\\\\\\\\\\\\\\\\\\\\\\\\\\\\\\\\\\\\\\\\\\\\\\\\\\\\\\\\\\\\\\\\\\\\\\\\\\\\\\\\\\\\\\\\\\\\\\\\\\\\\\\\\\\\\\\\\\

\section{Analysis of the results}

Trade-off analysis on the results obtained. Comparison with other protocols in terms of gain with security-cost trade-off. Observations and recommendations on the cost and security trade-off.\\\\\\\\\\\\\\\\\\\\\\\\\\\\\\\\\\\\\\\\\\\\\\\\\\\\\\\\\\\\\\\\\\\\\\\\\\\\\\\\\\\\\\\\\\\\\\\\\\\\\\\\\\\\\\\\\\\\\\\\\\\\\\\\\\\\\\\\\\\\\\\\\\\\\\\\\\\\\\\\\\\\\\\\\\\\\\\\\\\\\\\\\\\\\\\\\\\\\\\\\\\\\\\\\\\
\section{Conclusion}
Conclusion\\\\\\\\\\\\\\\\\\\\\\\\\\\\\\\\\\\\\\\\\\\\\\\\\\\\\\\\\\\\\
\bibliographystyle{abbrv}
\bibliography{}  % sigproc.bib is the name of the Bibliography in this case
% You must have a proper ".bib" file
%  and remember to run:
% latex bibtex latex latex
% to resolve all references
%
% ACM needs 'a single self-contained file'!
%
%APPENDICES are optional
%\balancecolumns
\balancecolumns
% That's all folks!
\end{document}
